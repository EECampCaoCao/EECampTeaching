\documentclass[12pt, a4paper]{article}

\usepackage[hmargin=2.5cm, vmargin=2cm]{geometry}
\usepackage{amsthm, amssymb, mathtools, yhmath, graphicx}
\usepackage{fontspec, type1cm, titlesec, titling, fancyhdr, tabularx}
\usepackage{pdftexcmds}
\usepackage{color, unicode-math, float, hhline, listings, minted}

\usepackage[CheckSingle, CJKmath]{xeCJK}
\usepackage{CJKulem}
\usepackage{enumitem}
\usepackage{tikz}
\usepackage{circuitikz}
%\setCJKmainfont[BoldFont=cwTex Q Hei]{cwTex Q Ming}
%\setCJKsansfont[BoldFont=cwTex Q Hei]{cwTex Q Ming}
\setmonofont{Source Code Pro}
\setCJKmainfont[BoldFont=cwTeX Q Hei]{cwTeX Q Ming}

\def\normalsize{\fontsize{12}{18}\selectfont}
\def\large{\fontsize{14}{21}\selectfont}
\def\Large{\fontsize{16}{24}\selectfont}
\def\LARGE{\fontsize{18}{27}\selectfont}
\def\huge{\fontsize{20}{30}\selectfont}

%\titleformat{\section}{\bf\Large}{\arabic{section}}{24pt}{}
%\titleformat{\subsection}{\large}{\arabic{subsection}.}{12pt}{}
%\titlespacing*{\subsection}{0pt}{0pt}{1.5ex}
\parindent=24pt

\DeclarePairedDelimiter{\abs}{\lvert}{\rvert}
\DeclarePairedDelimiter{\norm}{\lVert}{\rVert}
\DeclarePairedDelimiter{\inpd}{\langle}{\rangle}
\DeclarePairedDelimiter{\ceil}{\lceil}{\rceil}
\DeclarePairedDelimiter{\floor}{\lfloor}{\rfloor}

\newcommand{\img}{\mathsf{i}}
\newcommand{\ex}{\mathsf{e}}
\newcommand{\dD}{\mathrm{d}}
\newcommand{\dI}{\,\mathrm{d}}

\newlength{\mymathln}
\newcommand{\aligninside}[2]{
  \settowidth{\mymathln}{#2}
  \mathmakebox[\mymathln]{#1}
}

\setitemize{itemsep=0pt}
\setenumerate{itemsep=0pt}
\begin{document}

\section{四軸飛行器介紹}
\begin{figure}[H]
  \centering
  \includegraphics[width=0.4\textwidth]{Figures/quadcopter.jpg}
\caption{一個常見的四軸飛行器}
\label{fig:}
\end{figure}

\section{物理模型}
要維持四軸飛行器的平衡,須要考慮以下幾個物理量。
\begin{itemize}
  \item 外力造成的加速度,即 $a_x, a_y, a_z$ 。 
  \item 力矩造成的角加速度,即 $\alpha_x, \alpha_y, \alpha_z$ 。 
\end{itemize}
可以藉由控置馬達來條整四軸飛行器所受的力與力矩。
我們不妨假設四個馬達的推力分別為 $F_1, F_2, F_3, F_4$ 。

\subsection{升力}
四軸飛行器所受的升力即為四個馬達推力的總合,即
 \[ F = F_1 + F_2 + F_3 + F_4 \]

\subsection{力矩- $x, y$ 方向}
兩個對角的螺旋槳的升力差會使四軸飛行器受到一個 $x$ 或 $y$ 方向的力矩,
即
\[ \tau_x = (F_2 - F_4) r, \quad \tau_y = (F_3 - F_1) r \]
其中 $r$ 是 四軸的臂長(馬達到質心的距離)。

\subsection{力矩- $z$ 方向}
四軸飛行器的螺旋槳由1號到4號是正反轉交錯的,
而螺旋槳加速轉動時,或即使是等速轉動也會因
空氣的摩擦力,產生一個反方向的力矩。
 \[ \tau_z = F_1 - F_2 + F_3 - F_4 \]
 \begin{figure}[H]
  \centering
  \includegraphics[width=0.4\textwidth]{Figures/prop-dir.jpg}
\caption{}
\label{fig:}
\end{figure}

\section{PID}
四軸飛行器是一個不隱定的系統,我們必須藉助電腦控置來達到平衡。
PID 控置就是一種方法。 首先我們設定一個目標態,比如若要四軸飛行器
平衡,目標應該設為
\[ z = 0, \; \theta_x = \theta_y = \theta_z = 0 \]
對於上例的每一項,我們都有對應的控制方法。
比如若要 $z$ 增加,我們應該提高升力,即各個馬達的轉速。
而若要 $\theta_x$ 減少,我們便應該提高4號馬達的轉速,
降低2號馬達的轉速。 我們把這種調整稱作對應的反應(action)。
假設 $x$ 是目標的值, 而 $x'$ 是實際的值, 即誤差為
$e = x - x'$ 。 現在我們必須跟據誤差來調整反應
$y$ 的大小。

PID 就利用比例項(P), 積分項(I) 以及微分項 (D) 來調控反應。
公式為
\[ y(t) = K_p e(t) + K_i \int_{0}^{t} e(\tau) \dI \tau + K_d \frac{\dD e(t)}{\dD t}\]
其中 $K_p, K_i, K_d$ 分別為 比例項(P), 積分項(I) 以及微分項 (D) 的增益。
\subsection{比例項(P)}
比例項顧名思義,就是和誤差正比的那一項, 即 $K_p e(t)$ 。
$K_p$ 太低會使得系統反應太慢, 太高會使得系統不穩定。
\subsection{積分項(I)}
積分項可以消除系統的偏差,如馬達實際轉速不同等等。
$K_i$ 太大也會使得系統不穩定。
\subsection{微分項(D)}
微分項可以有效的使系統穩定,但容易受感應器的誤差所影響。

\subsection{離散PID}
因為我們運算、感應器偵測都是需要時間的,通常PID只能每 $T$ 單位的時間
才計算一次。 在這個情況下,
\[ y[n] = K_p e[n] + K_i \sum_0^{n} e[i] \Delta t_i+ K_d \frac{e[n] - e[n-1]}{\Delta t} \]

\section{實作細節}
\subsection{PID}
你必須實作 \texttt{mymath/pypid.py} 中的 \texttt{class MyPID} 。 
其中有兩個函數需要你修改 
\begin{itemize}
  \item \mintinline{python}{__init__(self, kp, ki, kd, *, imax)}:
    這個函數是用來初始化 PID 用的。 你可以在裡面宣告一些之後要
    用的變數。

  \item \mintinline{python}{get_control(self, t, err, derr)}:
    這個函數是主要的函數。 給定當前的時間 (\texttt{t}) 、 誤差
    (\texttt{err}) 以及誤要的微分 (\texttt{derr}) , 你必需回
    傳反應的大小。
\end{itemize}

\subsection{模擬器使用方法}
請在終端機下輸入 \mintinline{bash}{python3 main.py -s simple} 來執行模擬器。
在自動開起的瀏覽器按下 Start 鍵之後模擬器應該就會開始運行了。

\subsubsection{\texttt{\_\_init\_\_}}
\begin{minted}[linenos, frame=lines, framesep=2mm]{python3}
def __init__(self, kp, ki, kd, *, imax):
    super().__init__(kp, ki, kd, imax=imax)
    self.last_time = None
    # TODO:
    # Init the variables you need...
\end{minted}
這裡主要是放一些初始化需要做的事情。
\begin{itemize}
  \itemsep=0pt
  \item 第2行: 初始化這個物件所繼承的母物件。詳細的細節可以
    不用理會, 但要注意以下這些變數可用。
    \begin{itemize}
      \item \mintinline{python}{self.kp} : 即 $K_p$ 的值。
      \item \mintinline{python}{self.ki} : 即 $K_i$ 的值。
      \item \mintinline{python}{self.kd} : 即 $K_d$ 的值。
    \end{itemize}
  \item 第3..5行:
    這邊需要你初始化一些之後你所需要的變數。 請注意
    變數可能需要在 \mintinline{python}{self} 下定義,
    否則當這個函數結束後變數就會消失了。
    也就是說你的變數應該會長的像 \mintinline{python}{self.a}
    等等。 在這邊做為範例我們幫你初始化了一個變數
    \mintinline{python}{self.last_time} 為 
    \mintinline{python}{None} 了。
    至於要初始化哪些變數,可以接下來需要的時後再決定!
\end{itemize}
\subsubsection{\texttt{get\_control(self, t, err, derr)}}
\begin{minted}[linenos, frame=lines, framesep=2mm]{python3}
def get_control(self, t, err, derr):
    if self.last_time is None:
        self.last_time = t
        return 0.

    # TODO:
    # What should up be ?
    up = 0. 

    # TODO:
    # What should ud be ?
    ud = 0.

    # TODO:
    # Calculate dt. Remember to update
    # some variables you defined
    dt = 0.
    self.last_time = t

    # TODO:
    # Calculate the sum of the error.
    something = 0.

    # TODO:
    # Restrict the sum of the error to be within
    # [-self.int_restriction, self.int_restriction]
    if something > self.int_restriction:
        pass

    # TODO:
    # What should ui be ?
    ui = 0.

    # TODO:
    # Return the sum of up, ud and ui !
    return 0.
\end{minted}
程式碼雖然看起來很長,但其實不會很複雜! 我們一行一行來看!
\begin{itemize}
  \itemsep=0pt
  \item 第1行: 這個函數的參數。 第一個 \mintinline{python}{self} 
    顧名思意就是代表自己! \texttt{t}, \texttt{err}, \texttt{derr}
    則分別代表當前的時間、誤差以及微分向的誤差。 也就是
    \begin{align*}
      \texttt{t} &= t[n] \\
      \texttt{err} &= e[n] \\
      \texttt{derr} &\approx \frac{e[n] - e[n-1]}{t[n] - t[n-1]} \\
    \end{align*}
    至於為什麼 \texttt{derr} 明明可以用 \texttt{t}, \texttt{err} 得出卻
    還要傳一個近似的值進來呢? 其實是因為直接計算的話容易因為感測器誤差
    而出現不可遇期的結果,因此我們會用其他方法(Kalman Filter)幫你算好了。

  \item 第2..4行: 如果這是第一次執行 ($n = 0$) 我們直接回傳 $0$。
    這個部分全部幫你寫好了,不需要去動他。

  \item 第6..8行:
    還記得 PID 的公式嗎? 忘記了也沒關係! 我們再複習一次
    \begin{alignat*}{5}
      & y[n] &&= \;&&K_p e[n] &&+ K_i \sum_0^{n} e[i] \Delta t_i&&+ K_d \frac{e[n] - e[n-1]}{\Delta t} &&\\
      &    &&= \;&&u_p        &&+ u_i                         &&+ u_d &&\\
    \end{alignat*}
    其中 $\Delta t = t[n] - t[n-1]$ , 而 $u_p, u_i, u_d$ 分別是 P, I, D 項的反應。
    現在做為暖身請你修改第 $8$ 行的值,算出 $u_p$ (也就是 \texttt{up})。
    注意到 $e[n] = \texttt{err}$ , 且 $K_p$ 已經定義在 \mintinline{python}{self.kp} 了。

  \item 第10..12行:
    與剛剛類似,請修改第 $12$ 行的值,算出 $u_d$ (也就是 \texttt{ud})。

  \item 第14..18行:
    最後我們要開始算積分項。 首先我們要算出 $\Delta t$ (\texttt{dt})。
    請由 $\Delta t$ 的定義下手! 注意到 $t[n-1]$ 被存在 \mintinline{python}{self.last_time}
    裡了!\\
    算完了不要忘記更新 \mintinline{python}{self.last_time} 的值讓下一次計算可以使用!

  \item 第20..22行:
    計算積分的累計! 也就是要計算
    \[ I = \sum_0^n e[i] \Delta t_i \approx \sum_1^n e[i] \cdot (t[i] - t[i-1]) \]
    咦? 怎麼下標變成從 $1$ 開始了? 其實是因為 $n = 1$ 時 $t[i-1] = t[-1]$ 
    沒有定義。 這也是為什麼我們在 2..4 行要直接跳出的原因!

    至於要如何記算呢? 其實就是把 $e[n] \cdot \Delta t_n$ 一直累加而已。
    但是要累加在哪一個變數呢? 我們似乎還沒定義這樣的一個變數呢! \\
    嘿嘿! 這時候就要請你在 {\bf 初始化}的時候就定義好囉!

  \item 第24..28行:
    上一步我們計算出了 $I = \sum e[i] \Delta t_i$ , 但在實際中我們必須
    給這個數字一個閥值! 否則 $I$ 如果可以無限累積,最後一定會超出馬達可
    以容許的範圍! 

    在這裡我們會給他一個上限 $I_R$ ,也就是我們必須規定 $\abs{I} \leq I_R$ ,
    或是說 $-I_R \leq I \leq I_R$ 。 因此我們要做的事情就是
    \begin{itemize}
      \item Case 1: $-I_R \leq I \leq I_R$ 
            這個情況 $I$ 沒有超過範圍,所以我們不用動他。
      \item Case 2: $I \geq I_R$ 
            這個情況 $I$ 超過右界,所以我們把他拉回來,也就是令 $I = I_R$。
      \item Case 3: $I \leq -I_R$ 
            這個情況 $I$ 超過左界,所以同樣的我們令 $I = -I_R$。
    \end{itemize}

  \item 第30..32行:
    最後計算出 $u_i$ ,其實就是乘一個係數而已!

  \item 第30..32行:
    最後的最後別忘了把答案傳回去!
    否則就像考試忘了把答案填在答案卡上一樣!
  
\end{itemize}
\subsection{ 測試 }
如果你已經完成了以上程式,打開模擬器,理論上四軸飛行器應該可以保持平衡了!


\end{document}

